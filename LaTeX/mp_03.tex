%% Daniel Ellison 2019
%% LaTeX

\documentclass{hitec}

\usepackage{courier}
\usepackage{textcomp}

%% Python code style %%%%%%%%%%%%%%%%%%%%%%%%%%%%%%%%%%%%%%%%%%%%%%%%%%
\usepackage{listings}
\usepackage{textcomp}
\usepackage{xcolor}

\definecolor{grey}{RGB}{128,128,128}

\lstdefinestyle{custom}{%
        numberstyle=\tiny\color{grey},
        	basicstyle=\footnotesize\ttfamily,
        	captionpos=b,   
        numbers=left,      
        showstringspaces=false,
        numbersep=5pt,
        tabsize=4,
        upquote=true,
        frame = single
        }%
        
\lstset{%
	style=custom, 
	language=Python
	}%
%%%%%%%%%%%%%%%%%%%%%%%%%%%%%%%%%%%%%%%%%%%%%%%%%%%%%%%%%%%%%


\title{Mini Projects: Lecture 3}
\author{Matthew Ellison}
\company{MIT/SPISE}

\begin{document}

\maketitle



\section{Area Under the Curve}
\subsection{Part 1}
Suppose you want to find the area under the parabola $f(x)=x^2$ between 0 and $x=-5$ and 5. When we say area under the curve, we mean the area sandwiched between the curve and the x-axis, where the area counts as positive if it's above the x-axis and negative if it's below, as shown in the figure below. This is a standard problem of `integration' and can be solved exactly using the methods of calculus---but we can estimate the area very precisely using the computer, as you'll see.

One easy method to approximate the area is to first break the x-axis range of interest (here $-5$ to 5) into small pieces, say of size $.1$, and then make a rectangle that comes up at each section, and has height equal to the value of the parabola in the middle of the chunk, as shown in the image. The area we're looking for is then approximately equal to the total area of all the rectangles. Using smaller pieces would give a more accurate result.
Algebraically, if we let $x_0$, $x_1$, ..., $x_n$ be the corners of the rectangles on the x-axis in order (so $x_0=-5$, $x_n=5$), our approximation is to take Area $\approx(x_1-x_0)\times f\left(\frac{x_0+x_1}{2}\right)+(x_2-x_1)\times f\left(\frac{x_1+x_2}{2}\right)+...+(x_n-x_{n-1})\times f\left(\frac{x_n+x_{n-1}}{2}\right)$

Write a program to use this method to approximate the area of below the parabola mentioned above.

\subsection{Part 2}
Now let's make things a bit more general. Write a function which takes in the lower and upper bounds on $x$ (previously $-5$ and 5), and estimates the area under the parabola between the given x-values, using given step size (previously .1).

\subsection{Part 3}
Let's make things even more general! Extend your function from the previous part to take in any mathematical function, not just $f(x)=x^2$. A function may be passed as an argument in Python and may be called as shown in the example below:

\begin{lstlisting}
def evaluate_function_at_0(f):
	return f(0)
\end{lstlisting}

\noindent Your function should be exactly the same as before if the parabola is defined as a function like below, and then parabola is passed in for the function argument.

\begin{lstlisting}
def parabola(x):
	return x ** 2
\end{lstlisting}

\subsection{Extra Challenge}
The method used above, which estimates the area under the curve using rectangles---is improved upon by a technique known as Simpson's Method. Here's how it works:

\begin{enumerate}
	\item Break the x-values into pieces of some step size (such as .1). Call then $x_0$, $x_1$, $x_2$, ..., $x_n$ in increasing order with $x_0$ the left endpoint and $x_n$ the right endpoint. With Simpson's method, $n$ must be odd so that there are an even number of pieces in the x-range.
	\item Estimate the integral as $$\frac{step\_size}{3}(f(x_0)+4f(x_1)+2f(x_2)+4f(x_3)+...+2f(x_{n-2})+4f(x_{n-1})+f(x_n)).$$
\end{enumerate}

\noindent Write a function which takes as parameters a (mathematical) function, \texttt{step\_size}, left bound, and right bound and approximates the area under the curve using Simpson's method.

\end{document}