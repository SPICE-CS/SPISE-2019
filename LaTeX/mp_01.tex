%% Daniel Ellison 2019
%% LaTeX

\documentclass{hitec}

\usepackage{courier}
\usepackage{textcomp}

\title{Mini Projects: Lecture 1}
\author{Matthew Ellison}
\company{MIT/SPISE}

\begin{document}

\maketitle

\section{Counting Change}
Write a program which prompts the user to enter a number of pennies, nickels, dimes, and quarters. Then the program should print out the total value of all the coins.

\noindent If you need help getting started, here's an example of the functionality:

\begin{verbatim}
Welcome to Coin Counter!

How many pennies? 3 <---user entered 3, then pressed enter
How many nickels? 5
How many dimes? 2
How many quarters? 20

That comes to $5.48.	
\end{verbatim}

\section{Hashtag Border}
Write a program which prompts the user to enter a word. Then prints the word surrounded by hashtags, like this:

\begin{verbatim}
########
# aloe #
########
\end{verbatim}

\noindent \emph{\textbf{Hint 1:} You'll need to use the length command, \texttt{len()}, to find the appropriate number of hashtags for each line.}

\noindent \emph{\textbf{Hint 2:} \texttt{6*\textquotesingle s\textquotesingle} is the same as \texttt{\textquotesingle ssssss\textquotesingle}}

\section{Square Root}
Python has square root built in (e.g. \texttt{2 ** .5} gives \texttt{1.141...}), but you can write your own. Here's a common technique for finding the square root of some number $x$:

\begin{enumerate}
	\item Begin with initial guess \texttt{1}.
	\item Update your guess to be \texttt{.5 * (guess + x / guess)} .
	\item Repeat step 2 as needed.
	\item Return the value of guess as the square root.
\end{enumerate}

\noindent Write a program which prompts the user for the number x, performs the above procedure (maybe repeat step 2 three or so times), and then prints out the result---which should be close to the square root.

\section{Choose-Your-Own-Adventure}
Create a choose-your-own-adventure game, where the user is told a story and asked to make decisions along the way. This will put your \texttt{if}/\texttt{else} skills to the test! 

\noindent Here's an example beginning:

\begin{verbatim}	
Please Enter Your Name: Matt  <----user entered

Bad news, Matt.
You've woken up with a splitting headache. Fumbling in the dark, 
your hand brushes against a cold metallic object in the sand. 
Maybe it will be useful. Would you like to take it with 
you? [y/n] y <----user entered choice here

As your eyes adjust to the darkness, you notice outlines of 
a forest behind you--as well as a faint light further on 
the beach.Perhaps a distant campfire? 
Would you like to investigate? [y/n] n

[more story...]
\end{verbatim}

\end{document}